\documentclass[russian,14pt,a4paper,times]{extarticle}
\usepackage[T2A]{fontenc}
\usepackage[left=3cm, right=1cm, top=2cm, bottom=2.5cm]{geometry}
\usepackage[utf8x]{inputenc}
\usepackage{graphicx}
\usepackage{amssymb}
\usepackage{amsthm}
\usepackage{nameref}
\usepackage{babel}
\usepackage{indentfirst}
\usepackage{titlesec}
\usepackage{ragged2e}
\usepackage[none]{hyphenat}
\usepackage{tocloft}
\usepackage{titletoc}
\usepackage{caption}
\usepackage{tabularray}
\usepackage{setspace}
\usepackage{pgfplots}
\usepackage{chngcntr}
\graphicspath{ {images/} }
%\usepackage{hyperref}

%Создание команды для введения и заключения, по центру большими буквами
\newcommand*{\vved}[1]{
	\begin{center}
		\textbf{\MakeUppercase{#1}}
	\end{center}
	\addcontentsline{toc}{part}{#1}
}
%Создание команды для введения и заключения, по центру большими буквами

% Удаление сабсабсекций с содержания, если нужны, то закоментируйте
\setcounter{tocdepth}{2}
%\setcounter{secnumdepth}{2}
% Удаление сабсабсекций с содержания, если нужны, то закоментируйте

\sloppy
% Форматирование листа содержания
	\titlecontents{subsubsection}
[1.5em] %
{\smallskip}
{\thecontentslabel\hspace{1.02em}}%\thecontentslabel
{\hspace*{2.32em}}
{\,\,\titlerule*[0.77pc]{}\contentspage}

	\titlecontents{subsection}
[1.5em] %
{\smallskip}
{\thecontentslabel\hspace{1.02em}}%\thecontentslabel
{\hspace*{2.32em}}
{\,\,\titlerule*[0.77pc]{}\contentspage}

	\titlecontents{section}
[1.5em] %
{\smallskip}
{\thecontentslabel\hspace{1.80em}}%\thecontentslabel
{\hspace*{2.32em}}
{\,\,\titlerule*[0.77pc]{}\contentspage}

	\titlecontents{part}
[1.5em] %
{\smallskip}
{\thecontentslabel\hspace{1.80em}}%\thecontentslabel
{\hspace*{0em}}
{\,\,\titlerule*[0.77pc]{}\contentspage}
% Форматироваине листа содержания

% Содержание по центру и большими буквами
\renewcommand{\cfttoctitlefont}{\hfill\Large\fontsize{14}{16.8}\bfseries\MakeUppercase}
\renewcommand{\cftaftertoctitle}{\hfill\hfill}
\renewcommand{\baselinestretch}{1}
% Содержание по центру и большими буквами

% Установка размера и интервалов секций и подсекций и подподсекций
\titleformat{\section}{\fontsize{14}{16.8}\selectfont\bfseries}{\thesection}{0.5em}{}
\titleformat{\subsection}{\fontsize{14}{16.8}\selectfont\bfseries}{\thesubsection}{0.5em}{}
\titleformat{\subsubsection}{\fontsize{14}{16.8}\selectfont}{\thesubsubsection}{0.5em}{}
\titlespacing{\section}{\parindent}{0em}{1em}
\titlespacing{\subsection}{\parindent}{1.6em}{1em}
\titlespacing{\subsubsection}{\parindent}{1em}{1em}
% Установка размера и интервалов секций и подсекций подподсекций

%Установка подписи рисунков и таблиц по стандарту
\DeclareCaptionLabelSeparator{custom}{ -- }
\captionsetup[figure]{labelsep=custom, name=Рисунок, font=normalfont}
\captionsetup[table]{labelsep=custom, name=Таблица, font=normalfont}
%Установка подписи рисунков и таблиц по стандарту

% Установка нумерации рисунка и таблиц секция.номер
\counterwithin{figure}{section}
\renewcommand{\thefigure}{\arabic{section}.\arabic{figure}}
\counterwithin{table}{section}
\renewcommand{\thetable}{\arabic{section}.\arabic{table}}
% Установка нумерации рисунка и таблиц секция.номер




